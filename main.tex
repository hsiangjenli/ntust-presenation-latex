%----------------------------------------------------------------------------------------
%	PACKAGES AND THEMES
%----------------------------------------------------------------------------------------
\documentclass[aspectratio=169,xcolor=dvipsnames, t]{beamer}
\usepackage{fontspec} % Allows using custom font. MUST be before loading the theme!
\usetheme{SimplePlusAIC}
\usepackage{ulem} 
\usepackage{hhline}
\usepackage{hyperref}
\usepackage{graphicx} % Allows including images
\usepackage{booktabs} % Allows the use of \toprule, \midrule and  \bottomrule in tables
\usepackage{svg} %allows using svg figures
\usepackage{tikz}
\usepackage{makecell}
\usepackage{amsmath} % For \textsubscript command
\usepackage[english]{babel}
\usepackage[backend=biber, style=apa]{biblatex}
\addbibresource{test.bib}
\usepackage{wrapfig}
% ADD YOUR PACKAGES BELOW
\usetikzlibrary{calc,patterns.meta}
\usepackage{totcount}
\regtotcounter{section}
\usepackage{forest}
\usepackage{colortbl}
\usepackage{color,soul}
\usepackage{xcolor}
\usepackage{nameref}

% Redefinition of the \section command so that each one is labeled \label{sec:n} where n is its index 
\let\oldsection\section
\renewcommand{\section}[2][\relax]{%
    \ifx#1\relax
      \oldsection{#2}%
    \else
      \oldsection[#1]{#2}%
    \fi%
    \label{sec:\thesection}%
}

% Definition of custom colors based on the initial figure of the bar by the OP
\definecolor{myblue}{HTML}{005BAC}
\definecolor{mygreen}{HTML}{8BC53F}
\definecolor{mygray}{HTML}{DDDDDD}
\definecolor{BlueIuss}{HTML}{013B64}
\definecolor{LightBlueIuss}{HTML}{005BAC}

% Definition of custom tikz styles in order to ease readability
\tikzset{
    % Bar style (Argument : color)
    sectionbar/.style={
        % Filling with one color as a preaction, in order to avoid reset by the pattern color
        preaction={fill=#1!70},
        % Application of the line pattern on to of the fill
        pattern={Lines[angle=45,distance={6pt},line width=3pt]},pattern color=#1
    },
    % Node style (Arguments : color, section number)
    sectionnode/.style 2 args={
        fill=#1,
        draw=white,
        thick,
        circle,
        text=white,
        radius=10pt,
        % Display of the section name below the cicle
        label={[text=#1, text width=2cm, align=center]below:\nameref{sec:#2}},
        }
}


% Actual definition of the colorbar based on Gonzalo Medina's initial proposal
\makeatletter
    \def\pbar@progressbar{} % the progress bar
    \newcount\pbar@tmpcnta% auxiliary counter
    \newcount\pbar@tmpcntb% auxiliary counter
    \newdimen\pbar@pbht %progressbar height
    \newdimen\pbar@pbwd %progressbar width
    \newdimen\pbar@tmpdim % auxiliary dimension
    \pbar@pbwd=\linewidth
    \pbar@pbht=4pt

% The progress bar
\def\pbar@progressbar{%
    \pbar@tmpcnta=\value{section} % tmpcnta stores the section number
    \pbar@tmpcntb=\totvalue{section} % tmbcountb sotres the total amount of sections
    \advance\pbar@tmpcntb by 1 % tmbcountb is advanced by 1 in order to have the last bar segment after the last node

    \begin{tikzpicture}[very thin]
        % Clipping scope to avoid tests for the bar dimensions
        \begin{scope}
        % Clipping path
        \path[rounded corners=2pt,clip] (0pt,{-\pbar@pbht/2}) rectangle (\pbar@pbwd,{\pbar@pbht/2});
        % Gray bar (from 0 to last section)
        \path[sectionbar=mygray] (0pt,{-\pbar@pbht/2}) rectangle (\linewidth,{\pbar@pbht/2});
        % Blue bar (from 0 to the current section)
        \path[sectionbar=BlueIuss] (0pt,{-\pbar@pbht/2}) rectangle ({(\pbar@tmpcnta-0.5)*\linewidth/\pbar@tmpcntb},{\pbar@pbht/2});
        % Green bar (from current to next section)
        \path[sectionbar=LightBlueIuss] ({(\pbar@tmpcnta-0.5)*\linewidth/\pbar@tmpcntb},{-\pbar@pbht/2}) rectangle ({(\pbar@tmpcnta+0.5)*\linewidth/\pbar@tmpcntb},{\pbar@pbht/2});
        \end{scope}
        % Drawing of the nodes on top of the bars, based on the number of the current section
        \foreach \secnumber in {1,...,\totvalue{section}}{
            % Number is lower, section is past, blue color
            \ifnum\secnumber<\pbar@tmpcnta
                \node[sectionnode={BlueIuss}{\secnumber}] at ({(\secnumber-0.5)*\linewidth/\pbar@tmpcntb},0) {\strut\secnumber};
            \fi
            % Number is equal, section is current, green color
            \ifnum\secnumber=\pbar@tmpcnta
                \node[sectionnode={LightBlueIuss}{\secnumber}] at ({(\secnumber-0.5)*\linewidth/\pbar@tmpcntb},0) {\strut\secnumber};
            \fi
            % Number is larger, to be done section, gray color
            \ifnum\secnumber>\pbar@tmpcnta
            \node[sectionnode={mygray}{\secnumber}] at ({(\secnumber-0.5)*\linewidth/\pbar@tmpcntb},0) {\strut\secnumber};
            \fi
        }
  \end{tikzpicture}%
}

\addtobeamertemplate{headline}{}
{%
  \begin{beamercolorbox}[wd=\paperwidth,ht=12.2ex,center,dp=1ex]{white}%
    \pbar@progressbar%
  \end{beamercolorbox}%
}
\makeatother

%----------------------------------------------------------------------------------------
%	TITLE PAGE CONFIGURATION
%----------------------------------------------------------------------------------------

\title[short title]{Your presentation title} % The short title appears at the bottom of every slide, the full title is only on the title page
\subtitle{Subtitle}


\author{Author's name surname}
\institute[Short title]{Center for Neurocognition, Epistemology and Theoretical Syntax
\newline
University School for Advanced Studies IUSS Pavia
}
% Your institution as it will appear on the bottom of every slide, maybe shorthand to save space

\date{\today} % Date, can be changed to a custom date

%----------------------------------------------------------------------------------------
%	PRESENTATION SLIDES
%----------------------------------------------------------------------------------------

\makeatletter
\patchcmd{\beamer@sectionintoc}
  {\vfill}
  {\vskip\itemsep}
  {}
  {}
\makeatother  

\begin{document}
\maketitlepage
%\section{Overview}
\begin{frame}[t]{Overview}
    % Throughout your presentation, if you choose to use \section{} and \subsection{} commands, these will automatically be printed on this slide as an overview of your presentation    
    \tableofcontents 

\end{frame}

%-----------------------------------------------


% Section divider frame
\makesection{Basic blocks}

%------------------------------------------------
% Bullets
\begin{frame}{Bullet Points}
    \begin{itemize}
        \item Lorem ipsum dolor sit amet, consectetur adipiscing elit
        \item Aliquam blandit faucibus nisi, sit amet dapibus enim tempus eu
        \item Nulla commodo, erat quis gravida posuere, elit lacus lobortis est, quis porttitor odio mauris at libero
        \item Nam cursus est eget velit posuere pellentesque
        \item Vestibulum faucibus velit a augue condimentum quis convallis nulla gravida
        \item Nam cursus est eget velit posuere pellentesque
        \item Vestibulum faucibus velit a augue condimentum quis convallis nulla gravida
    \end{itemize}
\end{frame}

%------------------------------------------------
% Lists
\begin{frame}{Numbered List}
    \begin{enumerate}
        \item Lorem ipsum dolor sit amet, consectetur adipiscing elit
        \item Aliquam blandit faucibus nisi, sit amet dapibus enim tempus eu
        \item Nulla commodo, erat quis gravida posuere, elit lacus lobortis est, quis porttitor odio mauris at libero
        \item Nam cursus est eget velit posuere pellentesque
        \item Vestibulum faucibus velit a augue condimentum quis convallis nulla gravida
        \item Nam cursus est eget velit posuere pellentesque
        \item Vestibulum faucibus velit a augue condimentum quis convallis nulla gravida
    \end{enumerate}
\end{frame}

%------------------------------------------------
% Highlight boxes
\begin{frame}{Blocks of Highlighted Text}
    In this slide, some important text will be \alert{highlighted} because it's important. Please, don't abuse it.

    \begin{block}{Block}
        Sample text
    \end{block}

    \begin{alertblock}{Alertblock}
        Sample text in red box
    \end{alertblock}

    \begin{examples}
        Sample text in green box. The title of the block is ``Examples".
    \end{examples}
\end{frame}

%------------------------------------------------
% Double columns
\begin{frame}{Multiple Columns}
    \begin{columns}
    \begin{column}{0.45\textwidth}
      \colheader{Heading}
        \begin{enumerate}
            \item Statement
            \item Explanation
            \item Example
        \end{enumerate}
    \end{column}
    \begin{column}{0.45\textwidth}  %%<--- here
        \colheader{Heading}
        Lorem ipsum dolor sit amet, consectetur adipiscing elit. Integer lectus nisl, ultricies in feugiat rutrum, porttitor sit amet augue. Aliquam ut tortor mauris. Sed volutpat ante purus, quis accumsan dolor.
    \end{column}
    \end{columns}
\end{frame}
%------------------------------------------------
% Table
\begin{frame}{Table}
    \begin{table}
        \begin{tabular}{l l l}
            \toprule
            \textbf{Treatments} & \textbf{Response 1} & \textbf{Response 2} \\
            \midrule
            Treatment 1         & 0.0003262           & 0.562               \\
            Treatment 2         & 0.0015681           & 0.910               \\
            Treatment 3         & 0.0009271           & 0.296               \\
            \bottomrule
        \end{tabular}
        \caption{Table caption}
    \end{table}
\end{frame}
%------------------------------------------------
% Section divider frame
\makesection{Theorems and Figures}

%------------------------------------------------
% Theoerm (in highlighted box) and Equation in text
\begin{frame}{Theorem}
    \begin{theorem}[Mass--energy equivalence]
        $E = mc^2$
    \end{theorem}
    Equation in text
    \begin{equation}
        c^{2} = a^{2} + b^{2}
    \end{equation}
\end{frame}

%------------------------------------------------
% Figure without wrapped text
\begin{frame}{Figure}
    \begin{figure}
    \includegraphics[height=0.5\paperheight]{assets/ntust/ntust-logo-icon.png}
    \caption{Figure caption}
    \end{figure}
\end{frame}

%------------------------------------------------
% Figure with wrapped text
\begin{frame}{Wrapped Figure}
    \begin{wrapfigure}{r}{0.4\textwidth}
    \centering
    \includegraphics[width=0.4\textwidth]{assets/ntust/ntust-logo-icon.png}
    \caption{Figure caption}
\end{wrapfigure}
"Lorem ipsum dolor sit amet, consectetur adipiscing elit, sed do eiusmod tempor incididunt ut labore et dolore magna aliqua. Ut enim ad minim veniam, quis nostrud exercitation ullamco laboris nisi ut aliquip ex ea commodo consequat. Duis aute irure dolor in reprehenderit in voluptate velit esse cillum dolore eu fugiat nulla pariatur. Excepteur sint occaecat cupidatat non proident, sunt in culpa qui officia deserunt mollit anim id est laborum.Sed ut perspiciatis unde omnis iste natus error sit voluptatem accusantium doloremque laudantium, totam rem aperiam.
\end{frame}
%------------------------------------------------
% Section divider frame
\makesection{Citations and References}

%------------------------------------------------
% Citations
\begin{frame}[fragile] % 
    \frametitle{Citation}
    An example of the \verb|\cite| command to cite within the presentation:\\~

    This statement requires citation \cite{rizzi_1976}.
\end{frame}

%------------------------------------------------
% Refenrenced
\begin{frame}[allowframebreaks]{References}
    % Beamer does not support BibTeX so references must be inserted manually as below
   %\printbibliography[heading=none]
\end{frame}

%----------------------------------------------------------------------------------------
% Final PAGE
% Set the text that is showed on the final slide
\finalpagetext{Thank you for your attention}
%----------------------------------------------------------------------------------------
\makefinalpage
%----------------------------------------------------------------------------------------
\end{document}